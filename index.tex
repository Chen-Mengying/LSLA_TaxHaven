% Options for packages loaded elsewhere
% Options for packages loaded elsewhere
\PassOptionsToPackage{unicode}{hyperref}
\PassOptionsToPackage{hyphens}{url}
\PassOptionsToPackage{dvipsnames,svgnames,x11names}{xcolor}
%
\documentclass[
  letterpaper,
  DIV=11,
  numbers=noendperiod]{scrreprt}
\usepackage{xcolor}
\usepackage{amsmath,amssymb}
\setcounter{secnumdepth}{5}
\usepackage{iftex}
\ifPDFTeX
  \usepackage[T1]{fontenc}
  \usepackage[utf8]{inputenc}
  \usepackage{textcomp} % provide euro and other symbols
\else % if luatex or xetex
  \usepackage{unicode-math} % this also loads fontspec
  \defaultfontfeatures{Scale=MatchLowercase}
  \defaultfontfeatures[\rmfamily]{Ligatures=TeX,Scale=1}
\fi
\usepackage{lmodern}
\ifPDFTeX\else
  % xetex/luatex font selection
\fi
% Use upquote if available, for straight quotes in verbatim environments
\IfFileExists{upquote.sty}{\usepackage{upquote}}{}
\IfFileExists{microtype.sty}{% use microtype if available
  \usepackage[]{microtype}
  \UseMicrotypeSet[protrusion]{basicmath} % disable protrusion for tt fonts
}{}
\makeatletter
\@ifundefined{KOMAClassName}{% if non-KOMA class
  \IfFileExists{parskip.sty}{%
    \usepackage{parskip}
  }{% else
    \setlength{\parindent}{0pt}
    \setlength{\parskip}{6pt plus 2pt minus 1pt}}
}{% if KOMA class
  \KOMAoptions{parskip=half}}
\makeatother
% Make \paragraph and \subparagraph free-standing
\makeatletter
\ifx\paragraph\undefined\else
  \let\oldparagraph\paragraph
  \renewcommand{\paragraph}{
    \@ifstar
      \xxxParagraphStar
      \xxxParagraphNoStar
  }
  \newcommand{\xxxParagraphStar}[1]{\oldparagraph*{#1}\mbox{}}
  \newcommand{\xxxParagraphNoStar}[1]{\oldparagraph{#1}\mbox{}}
\fi
\ifx\subparagraph\undefined\else
  \let\oldsubparagraph\subparagraph
  \renewcommand{\subparagraph}{
    \@ifstar
      \xxxSubParagraphStar
      \xxxSubParagraphNoStar
  }
  \newcommand{\xxxSubParagraphStar}[1]{\oldsubparagraph*{#1}\mbox{}}
  \newcommand{\xxxSubParagraphNoStar}[1]{\oldsubparagraph{#1}\mbox{}}
\fi
\makeatother


\usepackage{longtable,booktabs,array}
\usepackage{calc} % for calculating minipage widths
% Correct order of tables after \paragraph or \subparagraph
\usepackage{etoolbox}
\makeatletter
\patchcmd\longtable{\par}{\if@noskipsec\mbox{}\fi\par}{}{}
\makeatother
% Allow footnotes in longtable head/foot
\IfFileExists{footnotehyper.sty}{\usepackage{footnotehyper}}{\usepackage{footnote}}
\makesavenoteenv{longtable}
\usepackage{graphicx}
\makeatletter
\newsavebox\pandoc@box
\newcommand*\pandocbounded[1]{% scales image to fit in text height/width
  \sbox\pandoc@box{#1}%
  \Gscale@div\@tempa{\textheight}{\dimexpr\ht\pandoc@box+\dp\pandoc@box\relax}%
  \Gscale@div\@tempb{\linewidth}{\wd\pandoc@box}%
  \ifdim\@tempb\p@<\@tempa\p@\let\@tempa\@tempb\fi% select the smaller of both
  \ifdim\@tempa\p@<\p@\scalebox{\@tempa}{\usebox\pandoc@box}%
  \else\usebox{\pandoc@box}%
  \fi%
}
% Set default figure placement to htbp
\def\fps@figure{htbp}
\makeatother





\setlength{\emergencystretch}{3em} % prevent overfull lines

\providecommand{\tightlist}{%
  \setlength{\itemsep}{0pt}\setlength{\parskip}{0pt}}



 


\KOMAoption{captions}{tableheading}
\makeatletter
\@ifpackageloaded{bookmark}{}{\usepackage{bookmark}}
\makeatother
\makeatletter
\@ifpackageloaded{caption}{}{\usepackage{caption}}
\AtBeginDocument{%
\ifdefined\contentsname
  \renewcommand*\contentsname{Table of contents}
\else
  \newcommand\contentsname{Table of contents}
\fi
\ifdefined\listfigurename
  \renewcommand*\listfigurename{List of Figures}
\else
  \newcommand\listfigurename{List of Figures}
\fi
\ifdefined\listtablename
  \renewcommand*\listtablename{List of Tables}
\else
  \newcommand\listtablename{List of Tables}
\fi
\ifdefined\figurename
  \renewcommand*\figurename{Figure}
\else
  \newcommand\figurename{Figure}
\fi
\ifdefined\tablename
  \renewcommand*\tablename{Table}
\else
  \newcommand\tablename{Table}
\fi
}
\@ifpackageloaded{float}{}{\usepackage{float}}
\floatstyle{ruled}
\@ifundefined{c@chapter}{\newfloat{codelisting}{h}{lop}}{\newfloat{codelisting}{h}{lop}[chapter]}
\floatname{codelisting}{Listing}
\newcommand*\listoflistings{\listof{codelisting}{List of Listings}}
\makeatother
\makeatletter
\makeatother
\makeatletter
\@ifpackageloaded{caption}{}{\usepackage{caption}}
\@ifpackageloaded{subcaption}{}{\usepackage{subcaption}}
\makeatother
\usepackage{bookmark}
\IfFileExists{xurl.sty}{\usepackage{xurl}}{} % add URL line breaks if available
\urlstyle{same}
\hypersetup{
  pdftitle={The Role of Tax Havens in Global Land Investments},
  pdfauthor={Mengying},
  colorlinks=true,
  linkcolor={blue},
  filecolor={Maroon},
  citecolor={Blue},
  urlcolor={Blue},
  pdfcreator={LaTeX via pandoc}}


\title{The Role of Tax Havens in Global Land Investments}
\author{Mengying}
\date{2025-11-11}
\begin{document}
\maketitle

\renewcommand*\contentsname{Table of contents}
{
\hypersetup{linkcolor=}
\setcounter{tocdepth}{2}
\tableofcontents
}

\bookmarksetup{startatroot}

\chapter*{Proposal Overview}\label{proposal-overview}
\addcontentsline{toc}{chapter}{Proposal Overview}

\markboth{Proposal Overview}{Proposal Overview}

This Quarto book presents the proposal draft for the research project on
\textbf{the role of tax havens in global land investments}, based on the
merged Land Matrix--ORBIS dataset.

The goal is to understand how offshore financial centres influence land
acquisition networks and their implications for transparency,
governance, and accountability.

\bookmarksetup{startatroot}

\chapter{Introduction}\label{introduction}

\begin{itemize}
\tightlist
\item
  \emph{Briefly introduce the financialisation of large-scale land
  acquisitions (LSLAs);}
\end{itemize}

\begin{itemize}
\tightlist
\item
  \emph{Explain why tax havens matter --- their role in concealing
  ownership and facilitating transnational investment flows;}
\end{itemize}

\begin{itemize}
\tightlist
\item
  \emph{Present the research objective: to analyze the scale, geography,
  and network position of tax haven investors and their link with
  negative impacts (conflicts, displacement, environmental
  degradation).}
\end{itemize}

Large-scale land acquisitions (LSLAs) have become deeply embedded in
global financial structures. Offshore jurisdictions --- commonly known
as \textbf{tax havens} --- play an increasingly central role by hosting
investment vehicles that channel international capital into agricultural
and resource-rich lands.

These offshore structures obscure ownership chains and make it difficult
to identify the \textbf{ultimate beneficial owners (UBOs)} behind major
land investments.

This research aims to contribute to the International Land Coalition
(ILC) and Land Matrix initiatives to enhance \textbf{transparency and
accountability} in land governance.

\bookmarksetup{startatroot}

\chapter{Data and Methodology}\label{data-and-methodology}

\begin{itemize}
\tightlist
\item
  \emph{\textbf{Data sources:} Land Matrix (deals), ORBIS (ownership
  structures), and a custom tax haven list (20 countries).}
\item
  \emph{\textbf{Sample:} Deals currently \texttt{in\ operation.}}
\item
  \emph{\textbf{Matching method:} Flag investors as
  \texttt{is\_tax\_haven\ =\ TRUE/FALSE} using ISO codes.}
\item
  \textbf{\emph{Analytical framework:}}

  \begin{itemize}
  \tightlist
  \item
    \emph{Distribution analysis (scale, time, and sectoral trends)}
  \item
    \emph{Network analysis (connectivity and centrality)}
  \item
    \emph{Role analysis (position in ownership chains)}
  \item
    \emph{Impact analysis (social and environmental risks)}
  \end{itemize}
\end{itemize}

\section{Data Preparation}\label{data-preparation}

\begin{itemize}
\tightlist
\item
  Select only ``in operation'' deals using
  \texttt{current\_implementation\_status}.
\item
  Match investor registration countries against tax haven lists (e.g.,
  Tax Haven Index
\item
  Create a binary variable \texttt{is\_tax\_haven\ =\ TRUE/FALSE}.
\end{itemize}

\subsection{Planned Outputs}\label{planned-outputs}

\begin{itemize}
\tightlist
\item
  Summary table of dataset size (deals, investors, tax haven share).
\item
  Count of unique investors and target countries.
\end{itemize}

\bookmarksetup{startatroot}

\chapter{Analysis}\label{analysis}

\section{Data Overview}\label{data-overview}

\subsection{\texorpdfstring{\textbf{Expected:}}{Expected:}}\label{expected}

\begin{itemize}
\item
  fig/text: Data coverage and share of ``in operation'' deals in the
  total deals.
\item
  xxxx unique investors identified, around xxxx\% of investors
  registered in tax havens.
\item
  fig: Temporal distribution of land deals
\end{itemize}

\section{Distribution Patterns}\label{distribution-patterns}

\begin{enumerate}
\def\labelenumi{\arabic{enumi}.}
\item
  bar: Share of tax haven vs.~non--tax haven investments
\item
  line: Evolution of tax haven investments over time -\textgreater{}
  Analysis of whether the use of tax havens has increased over time
\item
  bar: Top 10 tax haven investor countries -\textgreater{} Identify the
  primary source countries of tax havens
\item
  bar: Top 10 target countries for tax haven investors -\textgreater{}
  Identify the principal countries of investment
\item
  Map: Global investment flows from tax havens to target countries
  -\textgreater{} Visualising the global investment pathways of tax
  havens
\item
  bar: Tax haven investments by sector -\textgreater{} Comparing the
  proportion of tax havens across different investment sectors
\end{enumerate}

\section{Network Analysis}\label{network-analysis}

\begin{itemize}
\item
  Network graph

  \begin{itemize}
  \tightlist
  \item
    highlighting tax haven nodes
  \end{itemize}
\item
  Table of top 10 countries by centrality (degree, betweenness)

  \begin{itemize}
  \tightlist
  \item
    evaluate the bridging role of tax havens;
  \end{itemize}
\end{itemize}

\begin{enumerate}
\def\labelenumi{\arabic{enumi}.}
\item
  network map: Country-to-country investment network -\textgreater{}
  Illustrating the global investment structure and the locations of tax
  haven nodes
\item
  table: Top 10 countries by betweenness centrality -\textgreater{}
  Identifying which tax havens serve as key bridge nodes is crucial.
\item
  network map: Regional clusters of tax havens -\textgreater{}
  Demonstrating the clustering characteristics of Caribbean, European
  and Asian tax havens
\item
  line: Network density evolution of tax haven participation
  -\textgreater{} Monitor network connectivity trends over time
\end{enumerate}

\section{Role-Level Analysis}\label{role-level-analysis}

\begin{itemize}
\item
  Examine how tax haven entities appear across \textbf{investment
  roles}:

  \begin{itemize}
  \item
    Operating Company (OC)
  \item
    Shareholder (SH)
  \item
    Controlling Shareholder (CSH)
  \item
    Global Ultimate Owner (GUO)
  \end{itemize}
\item
  Summary table: Number of tax haven entities by role. -\textgreater{}
  Concentration of tax havens across OC, SH, CSH, and GUO tiers
\item
  Sankey diagram: Country → Role → Target country. -\textgreater{}
  Illustrating investment pathways and the role of tax havens within
  them
\end{itemize}

\section{Impact Assessment}\label{impact-assessment}

\bookmarksetup{startatroot}

\chapter{Discussion}\label{discussion}

\begin{itemize}
\item
  Discuss the intermediary role of tax havens in global land investment
  flows:

  \begin{itemize}
  \item
    Explain why financial capital relies on offshore jurisdictions
  \item
    How it challenges accountability;
  \end{itemize}
\item
  Highlight transparency and governance implications.
\end{itemize}

\bookmarksetup{startatroot}

\chapter{Conclusion and Next Steps}\label{conclusion-and-next-steps}

\begin{itemize}
\tightlist
\item
  Deeper study on network analysis
\end{itemize}

\bookmarksetup{startatroot}

\chapter*{References}\label{references}
\addcontentsline{toc}{chapter}{References}

\markboth{References}{References}

\phantomsection\label{refs}




\end{document}
